\documentclass[a4paper,10pt]{article}
\usepackage[]{graphicx}
\usepackage[]{times}
\usepackage{geometry}

\geometry{verbose,a4paper,tmargin=1.5cm,bmargin=2cm,lmargin=2cm,rmargin=2cm}
\renewcommand{\baselinestretch}{1.2}

\title{TER installation manual \\ \small (version 2.0 beta)}

\author{
GICI group \vspace{0.1cm} \\
\small Department of Information and Communications Engineering \\
\small Universitat Aut{\`o}noma Barcelona \\
\small http://www.gici.uab.es  -  http://www.gici.uab.es/TER \\
\small http://sourceforge.net/projects/ter \\
}

\date{November 2007}

\begin{document}
\maketitle

	TER is programmed using the JAVA language. The used compiler is the
	SUN JAVA 1.5. Hence, to run the application the Runtime Environment
	is necessary. Except for raw and pgm format,
	to load and save the different image types TER uses 
	the JAI (Java Advanced Imaging) library. Thus, if you need to
	manage different image formats apart of raw images you should
	install it. This software can be freely downloaded from
	\emph{http://java.sun.com}. 

	TER is provided with a single jar file (\emph{dist/TER.jar}), that
	contains the following applications: TERcode, TERdecode. To run one
	of these applications you can use the following command: 
	\emph{java-classpath dist/TERcode.java} and 
	\emph{java-classpath dist/TERdecode.java}
	applications. In a GNU/Linux environment you can also use the shell
	scripts \emph{TERcode}, \emph{TERdecode}, \ldots situated at the
	root of the TER directory. 

	To install TER as a system program you can follow the following
	steps: 

	\begin{enumerate}
		\item Decompress the TER distribution and locate the
		\emph{TERcode.jar}, \emph{TERcode.jar}, and \emph{TERdisplay.jar}
		files in some directory, for example in
		\emph{/usr/local/TER/TERcode.jar},
		\emph{/usr/local/TER/TERdecode.jar}, and
		\emph{/usr/local/TER/TERdisplay.jar}.
		\item Create the following shell script (for windows
		environments you can create a bat file): 
			\begin{verbatim}
			#!/bin/sh
			java -classpath /usr/local/TER/TERcode.jar TER.TERcode $@ 
			\end{verbatim}
			Rename this file and call it TERcode.
		\item Create another shell script (or bat file) for the decoder
		called \emph{TERdecode}.
		\item Create another shell script (or bat file) for the decoder
		called \emph{TERdisplay}. 
		\item Put all shell scripts in some directory included in your
		execution PATH variable. 
		\item To execute the encoder or decoder run \emph{TERcode},
		\emph{TERdecode}, or \emph{TERdisplay} from your shell. 
	\end{enumerate}

\end{document}
