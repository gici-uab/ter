\documentclass[a4paper,10pt]{article}
\usepackage[]{graphicx}
\usepackage[]{times}
\usepackage{geometry}

\geometry{verbose,a4paper,tmargin=1.5cm,bmargin=2cm,lmargin=2cm,rmargin=2cm}
\renewcommand{\baselinestretch}{1.0}

\title{TER user manual \\ \small (version 2.0 beta)}

\author{
GICI group \vspace{0.1cm} \\
\small Department of Information and Communications Engineering \\
\small Universitat Aut{\`o}noma Barcelona \\
\small http://www.gici.uab.es  -  http://www.gici.uab.es/TER \\
\small  http://sourceforge.net/projects/ter \\
}

\date{November 2007}

\begin{document}
\maketitle


%%%%%%%%%%%%%%%%%%%%%%%%%%%%%%%%%%%%%%%%%%%%%%%%%%%%%%%%%%%%%%%%%%%%%%%%%%%%%%%
\section{Overview}
\label{sect:overview}

	
TER is an implementation of the CCSDS Recommendation for Image Data
Compression (Recommended Standard CCSDS 122.0-B-1 Blue Book). TER
implements not only the CCSDS Recommendation, but also new features
add to the Recommended Standard. TER is designed and programmed with
the aim to provide a good basis to test and develop the CCSDS
Recommendation for Image Data Compression. The application provides
different functionalitites that can be controlled via its
parameters. These parameters can be passed as program arguments,
selecting some of the algorithms that the application
incorporates. More than 30 parameters allow the combination of
different processes to produce a compressed image codestream. Each one
of these parameters and its valid arguments are explained in the
application help and in this manual. When added features are used, the
encoded file is not compilant with the Recommendation. Parameters employed in the added
features are also required as program arguments to correctly perform
the decoding process.

TER also defines a new file syntax for the Recommendation by adding
some embedded headers.
The features added by the new file syntax are similar to the ones
available in JPEG2000 and in JPIP.
As in JPEG2000, the encoded stream can be ordered by: quality (L),
position (P), resolution (R), and component (C);
obtaining, among others, the following progression orders: LRCP, RLCP,
RPCL, PCRL, and CPRL. Embedded headers permit to decode in any progression
order, even if the image was encoded using a different one. In addition,
TER allows to decorrelate in the spectral domain
before encoding the whole image instead each band separately
as restricted in the Recommendation.

A development manual is also available and the API is well documented
to facilitate an easy understanding, extension and modification of
the application. All the design and implementation details are widely
explained in these manuals and the source code is commented. All
these information is publicly available in the web page
http://www.gici.uab.es/TER, where you can download the sources and
the manuals. To guarantee a free distribution, TER has the General
Public License (GNU GPL) of the Free Software Foundation
(http://www.fsf.org/licensing/licenses/gpl.html).


TER incorparates algorithms to make it flexible enough to manage
different kind of data, to control and monitor the compression 
stages, to extract statistical information, etc. As we explained
above TER is a platform to test new ideas, not to be a commercial
product or to be integrated to other applications that need support
for encoding/decoding files compilant with the CCSDS Recommendation
for Image Data Coding. The double functionality of the application
makes that the application performance cannot be as good as some other
implementations, so for commercial purposes the GICI group does not
recommend the use of this application.

We have tried to make a good design and efforts have been made to
develop and implement the application as useful as
desirable. However, we could not foresee all the needs of TER
users, so we will thank you for all the suggestions and comments
that you can report to us (gici-dev@abra.uab.es). 

\flushright{We hope you enjoy it, \vspace{0.25cm}\\GICI
  group}\flushleft{} 

\newpage
%%%%%%%%%%%%%%%%%%%%%%%%%%%%%%%%%%%%%%%%%%%%%%%%%%%%%%%%%%%%%%%%%%%%%%%%%%%%%%%
\section{Usage}
\label{sect:usage}

TER is divided in 3 applications: the encoder, the decoder
and, the extractor of images using the quality, postion,
component, and resolution scalability of a TER file. All
programs are encapsulated in jar files in the \emph{dist/}
directory and can be executed separately using the JVM. Each
application accepts its own parameters which can be passed as
simple command arguments, i.e. \emph{java -jar dist/TERcode.jar -i
  inputImage.pgm}. The distribution incorporates a shell script both 
for the encoder and the decoder called \emph{TERcode} and
\emph{TERdecode} to facilitate executions in a GNU/Linux
environment. TER may use a great amount of memory. Hence, it is
recommended to set the maximum amount of memory that the
application can allocate (usually the same as you computer RAM) via
the \emph{-Xmx} parameter of the JVM, i.e. \emph{java -Xmx512m -jar
  dist/TERcode.jar -i inputImage}.  


The usage of TER is very simple. To compress an image you can use
the encoder with some specific parameters. All the functionalities of
TER are explained with the application parameters, so 
it is recommended to read them to know what you can do. Do not
worry if you select incompatible parameters or functionalities; the
application detects these problems and displays warning messages to
the user. 

When you have a compressed codestream, you can use the decoder to
recover the original image and save it with other common
formats. TER decoder can manage codestreams compliant with the CCSDS
Recommendation for Image Data Coding generated by other
applications. However not all the options 
defined in the Recommended Standard are implemented. TER decoder can
also manage specific TER files not compilant with the Recommendation.
In such cases, it is required that the user indicates that the file is
not compilant and that passes to the decoder the parameters using by
the encoder as program parameters.

A typical use of the application is as follows: 

\hspace{0.5cm}\emph{\# TERcode -i lena.pgm -o lenaCompressed.ter} \\ 
\hspace{0.5cm}\emph{\# TERdecode -i lenaCompressed.ter -o
  lenaRecovered.pgm} \\  

or if you do not use the shell scripts:

\hspace{0.5cm}\emph{\# java -Xmx512m -jar dist/TERcode.jar -i
  lena.pgm -o lenaCompressed} \\ 
\hspace{0.5cm}\emph{\# java -Xmx512m -jar dist/TERdecode.jar -i
  lenaCompressed.ter -o lenaRecovered.pgm} \\  

\newpage
%%%%%%%%%%%%%%%%%%%%%%%%%%%%%%%%%%%%%%%%%%%%%%%%%%%%%%%%%%%%%%%%%%%%%%%%%%%%%%%
\section{Parameters}
\label{sect:parameters}

In this section all supported parameters both for the encoder and
the decoder are explained. This documentation is extracted from the
application help and it explains all the algorithms incorporated to
TER. By reading them you will know all the functionalities of the
application. 

Parameters have two formats: the long and the short
specification. Long specification has $-$$-$ at the beginning while
short specification has $-$ (it does not matter which one you
choose). Each parameter has its own arguments, which usually are
integers, floats, booleans (0 to indicate false and 1 to indicate
true) or strings. If the user specifies some invalid arguments, the
application will display warning messages. Most of these parameters
are not mandatory. When they are not specified default values are
used. The following table shows how each parameter will be
displayed in this manual: 

\begin{center}\begin{tabular}{|rr|rlrl|}
	 \hline
	 \multicolumn{2}{|l|}{\textbf{$-$$-$longParameter}} &
	 \multicolumn{4}{|l|}{$\{$parameter arguments$\}$} \\
	 \cline{3-6}
	 \multicolumn{2}{|l|}{\textbf{$-$shortParameter}} & \emph{Mandatory:} & Yes/No & &  \\
	 \hline
	 \emph{Explanation:} & \multicolumn{5}{|p{12cm}|}{Parameter explanation} \\
	 \hline
	 \emph{Default:} & \multicolumn{5}{|p{12cm}|}{Parameter default values.} \\
	 \hline
\end{tabular}\end{center}

\subsection{TERcoder parameters}
\label{sect:encoderParameters}

\begin{center}\begin{tabular}{|rr|rlrl|}
\hline
\multicolumn{2}{|l|}{\textbf{$-$$-$help}} & \multicolumn{4}{|l|}{} \\
\cline{3-6}
\multicolumn{2}{|l|}{\textbf{$-$h}} & \emph{Mandatory:} & No & &  \\
\hline
\emph{Explanation:} & \multicolumn{5}{|p{12cm}|}{Displays this help and exits program.} \\
\hline
\emph{Default:} & \multicolumn{5}{|p{12cm}|}{} \\
\hline
\end{tabular}\end{center}
\begin{center}\begin{tabular}{|rr|rlrl|}
\hline
\multicolumn{2}{|l|}{\textbf{$-$$-$inputImage}} & \multicolumn{4}{|l|}{$\{$string$\}$} \\
\cline{3-6}
\multicolumn{2}{|l|}{\textbf{$-$i}} & \emph{Mandatory:} & Yes & &  \\
\hline
\emph{Explanation:} & \multicolumn{5}{|p{12cm}|}{Input image. Valid formats are: pgm, ppm, pbm, jpg, tiff, png, bmp, gif, fpx. If image is raw data file extension must be ".raw" or ".img" and "-g" parameter is mandatory.} \\
\hline
\emph{Default:} & \multicolumn{5}{|p{12cm}|}{} \\
\hline
\end{tabular}\end{center}
\begin{center}\begin{tabular}{|rr|rlrl|}
\hline
\multicolumn{2}{|l|}{\textbf{$-$$-$imageGeometry}} & \multicolumn{4}{|l|}{$\{$int int int int boolean$\}$} \\
\cline{3-6}
\multicolumn{2}{|l|}{\textbf{$-$g}} & \emph{Mandatory:} & No & &  \\
\hline
\emph{Explanation:} & \multicolumn{5}{|p{12cm}|}{Geometry of raw image data. Parameters are:\newline 1- zSize (number of image components)\newline 2- ySize (image height)\newline 3- xSize (image width)\newline 4- data type. Possible values are:\newline \hspace*{0.5truecm} 0- boolean (1 byte)\newline \hspace*{0.5truecm} 1- unsigned int (1 byte)\newline \hspace*{0.5truecm} 2- unsigned int (2 bytes)\newline \hspace*{0.5truecm} 3- signed int (2 bytes)\newline \hspace*{0.5truecm} 4- signed int (4 bytes)\newline \hspace*{0.5truecm} 5- signed int (8 bytes)\newline \hspace*{0.5truecm} 6- float (4 bytes)\newline \hspace*{0.5truecm} 7- double (8 bytes)\newline 5- Byte order (0 if BIG ENDIAN, 1 if LITTLE ENDIAN)\newline 6- 1 if 3 first components are RGB, 0 otherwise.} \\
\hline
\emph{Default:} & \multicolumn{5}{|p{12cm}|}{} \\
\hline
\end{tabular}\end{center}
\begin{center}\begin{tabular}{|rr|rlrl|}
\hline
\multicolumn{2}{|l|}{\textbf{$-$$-$outputFile}} & \multicolumn{4}{|l|}{$\{$string$\}$} \\
\cline{3-6}
\multicolumn{2}{|l|}{\textbf{$-$o}} & \emph{Mandatory:} & No & &  \\
\hline
\emph{Explanation:} & \multicolumn{5}{|p{12cm}|}{Output image file name (specified without extension).} \\
\hline
\emph{Default:} & \multicolumn{5}{|p{12cm}|}{same as input with a new extension (see "-of" parameter)} \\
\hline
\end{tabular}\end{center}
\begin{center}\begin{tabular}{|rr|rlrl|}
\hline
\multicolumn{2}{|l|}{\textbf{$-$$-$outputFileType}} & \multicolumn{4}{|l|}{$\{$int$\}$} \\
\cline{3-6}
\multicolumn{2}{|l|}{\textbf{$-$of}} & \emph{Mandatory:} & No & &  \\
\hline
\emph{Explanation:} & \multicolumn{5}{|p{12cm}|}{File type to generate. Valid values are:\newline 0- CCSDS Recommended file (.rec) \newline 1- TER file (.ter)} \\
\hline
\emph{Default:} & \multicolumn{5}{|p{12cm}|}{0 } \\
\hline
\end{tabular}\end{center}
\begin{center}\begin{tabular}{|rr|rlrl|}
\hline
\multicolumn{2}{|l|}{\textbf{$-$$-$part2Flag}} & \multicolumn{4}{|l|}{$\{$int$[$int$[$ int$[$ ...$]$$]$$]$$\}$} \\
\cline{3-6}
\multicolumn{2}{|l|}{\textbf{$-$h2}} & \emph{Mandatory:} & No & &  \\
\hline
\emph{Explanation:} & \multicolumn{5}{|p{12cm}|}{Indicates for each segment the presence of Part 2 header.\newline 0 - Part 2 header absent\newline 1 - Part 2 header present} \\
\hline
\emph{Default:} & \multicolumn{5}{|p{12cm}|}{1 } \\
\hline
\end{tabular}\end{center}
\begin{center}\begin{tabular}{|rr|rlrl|}
\hline
\multicolumn{2}{|l|}{\textbf{$-$$-$part3Flag}} & \multicolumn{4}{|l|}{$\{$int$[$int$[$ int$[$ ...$]$$]$$]$$\}$} \\
\cline{3-6}
\multicolumn{2}{|l|}{\textbf{$-$h3}} & \emph{Mandatory:} & No & &  \\
\hline
\emph{Explanation:} & \multicolumn{5}{|p{12cm}|}{Indicates for each segment the presence of Part 3 header.\newline 0 - Part 3 header absent\newline 1 - Part 3 header present} \\
\hline
\emph{Default:} & \multicolumn{5}{|p{12cm}|}{1 } \\
\hline
\end{tabular}\end{center}
\begin{center}\begin{tabular}{|rr|rlrl|}
\hline
\multicolumn{2}{|l|}{\textbf{$-$$-$part4Flag}} & \multicolumn{4}{|l|}{$\{$int$[$int$[$ int$[$ ...$]$$]$$]$$\}$} \\
\cline{3-6}
\multicolumn{2}{|l|}{\textbf{$-$h4}} & \emph{Mandatory:} & No & &  \\
\hline
\emph{Explanation:} & \multicolumn{5}{|p{12cm}|}{Indicates for each segment the presence of Part 4 header.\newline 0 - Part 4 header absent\newline 1 - Part 4 header present} \\
\hline
\emph{Default:} & \multicolumn{5}{|p{12cm}|}{1 } \\
\hline
\end{tabular}\end{center}
\begin{center}\begin{tabular}{|rr|rlrl|}
\hline
\multicolumn{2}{|l|}{\textbf{$-$$-$segByteLimit}} & \multicolumn{4}{|l|}{$\{$int$[$ int$[$ int$[$ ...$]$$]$$]$$\}$} \\
\cline{3-6}
\multicolumn{2}{|l|}{\textbf{$-$bl}} & \emph{Mandatory:} & No & &  \\
\hline
\emph{Explanation:} & \multicolumn{5}{|p{12cm}|}{Maximum number of bytes that can be used in a segment.\newline The value of SegByteLimits includes bytes used for the header, and is applied to the subsequent segments until a new value is given. Value should be in the interval (0 , 134217728] and it is expressed mod(134217728)} \\
\hline
\emph{Default:} & \multicolumn{5}{|p{12cm}|}{134217728 } \\
\hline
\end{tabular}\end{center}
\begin{center}\begin{tabular}{|rr|rlrl|}
\hline
\multicolumn{2}{|l|}{\textbf{$-$$-$DCStop}} & \multicolumn{4}{|l|}{$\{$int$[$ int$[$ int$[$ ...$]$$]$$]$$\}$} \\
\cline{3-6}
\multicolumn{2}{|l|}{\textbf{$-$dc}} & \emph{Mandatory:} & No & &  \\
\hline
\emph{Explanation:} & \multicolumn{5}{|p{12cm}|}{Indicates whether compressed output stops after coding of quantized DC coefficients.  If only one value is specified, type will be used for all image segments.} \\
\hline
\emph{Default:} & \multicolumn{5}{|p{12cm}|}{0 } \\
\hline
\end{tabular}\end{center}
\begin{center}\begin{tabular}{|rr|rlrl|}
\hline
\multicolumn{2}{|l|}{\textbf{$-$$-$bitPlaneStop}} & \multicolumn{4}{|l|}{$\{$int$[$ int$[$ int$[$ ...$]$$]$$]$$\}$} \\
\cline{3-6}
\multicolumn{2}{|l|}{\textbf{$-$bp}} & \emph{Mandatory:} & No & &  \\
\hline
\emph{Explanation:} & \multicolumn{5}{|p{12cm}|}{Only used when DCStop equals 0. Indicates the bit plane index where the codification should stop.  If only one value is specified, type will be used for all image segments.} \\
\hline
\emph{Default:} & \multicolumn{5}{|p{12cm}|}{0 } \\
\hline
\end{tabular}\end{center}
\begin{center}\begin{tabular}{|rr|rlrl|}
\hline
\multicolumn{2}{|l|}{\textbf{$-$$-$stageStop}} & \multicolumn{4}{|l|}{$\{$int$[$ int$[$ int$[$ ...$]$$]$$]$$\}$} \\
\cline{3-6}
\multicolumn{2}{|l|}{\textbf{$-$ss}} & \emph{Mandatory:} & No & &  \\
\hline
\emph{Explanation:} & \multicolumn{5}{|p{12cm}|}{Indicates the stage where the codification ends.  If only one value is specified, type will be used for all image segments. If three levels of DWT are applied, then: \newline 1 - stage 1 \newline 2 - stage 2 \newline 3 - stage 3 \newline 4 - stage 4 } \\
\hline
\emph{Default:} & \multicolumn{5}{|p{12cm}|}{The same of wavelet levels plus one} \\
\hline
\end{tabular}\end{center}
\begin{center}\begin{tabular}{|rr|rlrl|}
\hline
\multicolumn{2}{|l|}{\textbf{$-$$-$useFill}} & \multicolumn{4}{|l|}{$\{$int$[$ int$[$ int$[$ ...$]$$]$$]$$\}$} \\
\cline{3-6}
\multicolumn{2}{|l|}{\textbf{$-$uf}} & \emph{Mandatory:} & No & &  \\
\hline
\emph{Explanation:} & \multicolumn{5}{|p{12cm}|}{Specifies whether fill bits will be used to produce SegByteLimit bytes in each segment.  If only one value is specified, type will be used for all image segments.} \\
\hline
\emph{Default:} & \multicolumn{5}{|p{12cm}|}{0 } \\
\hline
\end{tabular}\end{center}
\begin{center}\begin{tabular}{|rr|rlrl|}
\hline
\multicolumn{2}{|l|}{\textbf{$-$$-$blocksPerSegment}} & \multicolumn{4}{|l|}{$\{$int$[$ int$[$ int$[$ ...$]$$]$$]$$\}$} \\
\cline{3-6}
\multicolumn{2}{|l|}{\textbf{$-$bs}} & \emph{Mandatory:} & No & &  \\
\hline
\emph{Explanation:} & \multicolumn{5}{|p{12cm}|}{Indicates the number of blocks contained in a segment. If only one value is specified, type will be used for all image segments.} \\
\hline
\emph{Default:} & \multicolumn{5}{|p{12cm}|}{1048576 } \\
\hline
\end{tabular}\end{center}
\begin{center}\begin{tabular}{|rr|rlrl|}
\hline
\multicolumn{2}{|l|}{\textbf{$-$$-$optDCSelect}} & \multicolumn{4}{|l|}{$\{$int$[$ int$[$ int$[$ ...$]$$]$$]$$\}$} \\
\cline{3-6}
\multicolumn{2}{|l|}{\textbf{$-$ds}} & \emph{Mandatory:} & No & &  \\
\hline
\emph{Explanation:} & \multicolumn{5}{|p{12cm}|}{Indicates the method used to select k parameter while coding DC components} \\
\hline
\emph{Default:} & \multicolumn{5}{|p{12cm}|}{1 } \\
\hline
\end{tabular}\end{center}
\begin{center}\begin{tabular}{|rr|rlrl|}
\hline
\multicolumn{2}{|l|}{\textbf{$-$$-$optACSelect}} & \multicolumn{4}{|l|}{$\{$int$[$ int$[$ int$[$ ...$]$$]$$]$$\}$} \\
\cline{3-6}
\multicolumn{2}{|l|}{\textbf{$-$as}} & \emph{Mandatory:} & No & &  \\
\hline
\emph{Explanation:} & \multicolumn{5}{|p{12cm}|}{Indicates the method used to select k parameter while coding BitDepthAC\_Block components} \\
\hline
\emph{Default:} & \multicolumn{5}{|p{12cm}|}{1 } \\
\hline
\end{tabular}\end{center}
\begin{center}\begin{tabular}{|rr|rlrl|}
\hline
\multicolumn{2}{|l|}{\textbf{$-$$-$WTType}} & \multicolumn{4}{|l|}{$\{$int$[$ int$[$ int$[$ ...$]$$]$$]$$\}$} \\
\cline{3-6}
\multicolumn{2}{|l|}{\textbf{$-$wt}} & \emph{Mandatory:} & No & &  \\
\hline
\emph{Explanation:} & \multicolumn{5}{|p{12cm}|}{Discrete wavelet transform type for each image component in the spatial domain. First value is for the first component, second value for the second component and so on. If only one value is specified, wavelete transform type will be the same for all components. Valid values are:\newline 0 - No wavelet transform \newline 1 - Integer (Reversible) 5/3 DWT \newline 2 - Real Isorange (irreversible) 9/7 DWT (JPEG2000 standard) \newline 3 - Real Isonorm (irreversible) 9/7 DWT (CCSDS-Recommended) \newline 4 - Integer (Reversible) 9/7M DWT (CCSDS-Recommended) \newline 5 - Integer 5/3 DWT (classic construction) \newline 6 - Integer 9/7 DWT (classic construction)} \\
\hline
\emph{Default:} & \multicolumn{5}{|p{12cm}|}{4 } \\
\hline
\end{tabular}\end{center}
\begin{center}\begin{tabular}{|rr|rlrl|}
\hline
\multicolumn{2}{|l|}{\textbf{$-$$-$signedPixels}} & \multicolumn{4}{|l|}{$\{$int$[$ int$[$ int$[$ ...$]$$]$$]$$\}$} \\
\cline{3-6}
\multicolumn{2}{|l|}{\textbf{$-$sp}} & \emph{Mandatory:} & No & &  \\
\hline
\emph{Explanation:} & \multicolumn{5}{|p{12cm}|}{For each segment (entire image image in the Recommendation) specifies if the input pixels are signed or unsigned quantities. Valid values are:\newline 0 - unsigned \newline 1 - signed} \\
\hline
\emph{Default:} & \multicolumn{5}{|p{12cm}|}{0 } \\
\hline
\end{tabular}\end{center}
\begin{center}\begin{tabular}{|rr|rlrl|}
\hline
\multicolumn{2}{|l|}{\textbf{$-$$-$transposeImg}} & \multicolumn{4}{|l|}{$\{$int$[$ int$[$ int$[$ ...$]$$]$$]$$\}$} \\
\cline{3-6}
\multicolumn{2}{|l|}{\textbf{$-$ti}} & \emph{Mandatory:} & No & &  \\
\hline
\emph{Explanation:} & \multicolumn{5}{|p{12cm}|}{Indicates whether the channel (entire image in the Recommendation) should be transposed after reconstruction. Valid values are:\newline 0 - do not transpose image \newline 1 - transpose image} \\
\hline
\emph{Default:} & \multicolumn{5}{|p{12cm}|}{0 } \\
\hline
\end{tabular}\end{center}
\begin{center}\begin{tabular}{|rr|rlrl|}
\hline
\multicolumn{2}{|l|}{\textbf{$-$$-$codeWordLength}} & \multicolumn{4}{|l|}{$\{$int$[$ int$[$ int$[$ ...$]$$]$$]$$\}$} \\
\cline{3-6}
\multicolumn{2}{|l|}{\textbf{$-$cl}} & \emph{Mandatory:} & No & &  \\
\hline
\emph{Explanation:} & \multicolumn{5}{|p{12cm}|}{Indicated the coded word length for each segment. Valid values are:\newline 0 - 8-bit word \newline 1 - 16-bit word \newline 2 - 24-bit word \newline 3 - 32-bit word} \\
\hline
\emph{Default:} & \multicolumn{5}{|p{12cm}|}{0 } \\
\hline
\end{tabular}\end{center}
\begin{center}\begin{tabular}{|rr|rlrl|}
\hline
\multicolumn{2}{|l|}{\textbf{$-$$-$customWtFlag}} & \multicolumn{4}{|l|}{$\{$int$[$ int$[$ int$[$ ...$]$$]$$]$$\}$} \\
\cline{3-6}
\multicolumn{2}{|l|}{\textbf{$-$wg}} & \emph{Mandatory:} & No & &  \\
\hline
\emph{Explanation:} & \multicolumn{5}{|p{12cm}|}{Weighting type for each component. First value is for the first component, second for the second component and so on. If only one value is specified, type will be used for all image components.\newline 0 - CCSDS recommended\newline 1 - Defined by the user\newline 2 - No weighting} \\
\hline
\emph{Default:} & \multicolumn{5}{|p{12cm}|}{0 } \\
\hline
\end{tabular}\end{center}
\begin{center}\begin{tabular}{|rr|rlrl|}
\hline
\multicolumn{2}{|l|}{\textbf{$-$$-$customWeight}} & \multicolumn{4}{|l|}{$\{$float$[$ float$[$ float$[$ ...$]$$]$$]$$\}$} \\
\cline{3-6}
\multicolumn{2}{|l|}{\textbf{$-$cw}} & \emph{Mandatory:} & No & &  \\
\hline
\emph{Explanation:} & \multicolumn{5}{|p{12cm}|}{Custom weights defined by the user. For each subband of each component a weighting factor should be given for each subband. If only the number of parameters introduced is not correct the programm stops. The order of the weights is as follows: For each component :\newline LL\_n,HL\_n,LH\_n,HH\_n,HL\_n-1,LH\_n-1,HH\_n-1,...,HL\_0,LH\_0,HH\_0.} \\
\hline
\emph{Default:} & \multicolumn{5}{|p{12cm}|}{} \\
\hline
\end{tabular}\end{center}
\begin{center}\begin{tabular}{|rr|rlrl|}
\hline
\multicolumn{2}{|l|}{\textbf{$-$$-$WTLevels}} & \multicolumn{4}{|l|}{$\{$int$[$ int$[$ int$[$ ...$]$$]$$]$$\}$} \\
\cline{3-6}
\multicolumn{2}{|l|}{\textbf{$-$wl}} & \emph{Mandatory:} & No & &  \\
\hline
\emph{Explanation:} & \multicolumn{5}{|p{12cm}|}{Discrete wavelet transform levels for each image component in the spatial domain. First value is for the first component, second value for the second component and so on. If only one value is specified, wavelet transform levels will be the same for all components.} \\
\hline
\emph{Default:} & \multicolumn{5}{|p{12cm}|}{3 } \\
\hline
\end{tabular}\end{center}
\begin{center}\begin{tabular}{|rr|rlrl|}
\hline
\multicolumn{2}{|l|}{\textbf{$-$$-$imageExtensionType}} & \multicolumn{4}{|l|}{$\{$int$[$ int$[$ int$[$ ...$]$$]$$]$$\}$} \\
\cline{3-6}
\multicolumn{2}{|l|}{\textbf{$-$ie}} & \emph{Mandatory:} & No & &  \\
\hline
\emph{Explanation:} & \multicolumn{5}{|p{12cm}|}{Indicates the kind of extension that has been applied to the each component of the image in order to make it able to be compressed using TER.  If only one value is specified, image extension type will be the same for all components. Valid values are:\newline 0 - Repeating last value\newline 1 - Symmetric expansion\newline 2 - No extension} \\
\hline
\emph{Default:} & \multicolumn{5}{|p{12cm}|}{0 } \\
\hline
\end{tabular}\end{center}
\begin{center}\begin{tabular}{|rr|rlrl|}
\hline
\multicolumn{2}{|l|}{\textbf{$-$$-$WTOrder}} & \multicolumn{4}{|l|}{$\{$int$[$ int$[$ int$[$ ...$]$$]$$]$$\}$} \\
\cline{3-6}
\multicolumn{2}{|l|}{\textbf{$-$wo}} & \emph{Mandatory:} & No & &  \\
\hline
\emph{Explanation:} & \multicolumn{5}{|p{12cm}|}{Order in which the Discrete wavelet transform is performed over the spatial dimentions. First value is for the first component, second value for the second component and so on. If only one value is specified, wavelet transform order will be the same for all components. \newline 0 - Horizontal - Vertical \newline 1 - Vertical - Horizontal \newline 2 - Only horizontal} \\
\hline
\emph{Default:} & \multicolumn{5}{|p{12cm}|}{0 } \\
\hline
\end{tabular}\end{center}
\begin{center}\begin{tabular}{|rr|rlrl|}
\hline
\multicolumn{2}{|l|}{\textbf{$-$$-$gaggleDCSize}} & \multicolumn{4}{|l|}{$\{$int$[$ int$[$ int$[$ ...$]$$]$$]$$\}$} \\
\cline{3-6}
\multicolumn{2}{|l|}{\textbf{$-$gd}} & \emph{Mandatory:} & No & &  \\
\hline
\emph{Explanation:} & \multicolumn{5}{|p{12cm}|}{Size of a gaggle for DC components in each segment of the image. First value is for the first segment, second value for the second segment and so on. If only one value is specified, the size of DC gaggles will be the same for all segments. 0 value means that all the blocks of the segment are in the same gaggle} \\
\hline
\emph{Default:} & \multicolumn{5}{|p{12cm}|}{16 } \\
\hline
\end{tabular}\end{center}
\begin{center}\begin{tabular}{|rr|rlrl|}
\hline
\multicolumn{2}{|l|}{\textbf{$-$$-$gaggleACSize}} & \multicolumn{4}{|l|}{$\{$int$[$ int$[$ int$[$ ...$]$$]$$]$$\}$} \\
\cline{3-6}
\multicolumn{2}{|l|}{\textbf{$-$ga}} & \emph{Mandatory:} & No & &  \\
\hline
\emph{Explanation:} & \multicolumn{5}{|p{12cm}|}{Size of a gaggle for AC components in each segment of the image. First value is for the first segment, second value for the second segment and so on. If only one value is specified, the size of AC gaggles will be the same for all segments. 0 value means that all the blocks of the segment are in the same gaggle} \\
\hline
\emph{Default:} & \multicolumn{5}{|p{12cm}|}{16 } \\
\hline
\end{tabular}\end{center}
\begin{center}\begin{tabular}{|rr|rlrl|}
\hline
\multicolumn{2}{|l|}{\textbf{$-$$-$idDC}} & \multicolumn{4}{|l|}{$\{$int$[$ int$[$ int$[$ ...$]$$]$$]$$\}$} \\
\cline{3-6}
\multicolumn{2}{|l|}{\textbf{$-$id}} & \emph{Mandatory:} & No & &  \\
\hline
\emph{Explanation:} & \multicolumn{5}{|p{12cm}|}{Number which indicates how often the ID DC appears. First value is for the first segment, second value for the second segment and so on. If only one value is specified, the frequency of ID DC will be the same for all segments. 0 value means that only an ID DC is used for the segment} \\
\hline
\emph{Default:} & \multicolumn{5}{|p{12cm}|}{0 } \\
\hline
\end{tabular}\end{center}
\begin{center}\begin{tabular}{|rr|rlrl|}
\hline
\multicolumn{2}{|l|}{\textbf{$-$$-$idAC}} & \multicolumn{4}{|l|}{$\{$int$[$ int$[$ int$[$ ...$]$$]$$]$$\}$} \\
\cline{3-6}
\multicolumn{2}{|l|}{\textbf{$-$ia}} & \emph{Mandatory:} & No & &  \\
\hline
\emph{Explanation:} & \multicolumn{5}{|p{12cm}|}{Number which indicates how often the ID AC appears. First value is for the first segment, second value for the second segment and so on. If only one value is specified, the frequency of ID AC will be the same for all segments. 0 value means that only an ID DC is used for the segment} \\
\hline
\emph{Default:} & \multicolumn{5}{|p{12cm}|}{0 } \\
\hline
\end{tabular}\end{center}
\begin{center}\begin{tabular}{|rr|rlrl|}
\hline
\multicolumn{2}{|l|}{\textbf{$-$$-$entropyAC}} & \multicolumn{4}{|l|}{$\{$int$[$ int$[$ int$[$ ...$]$$]$$]$$\}$} \\
\cline{3-6}
\multicolumn{2}{|l|}{\textbf{$-$ea}} & \emph{Mandatory:} & No & &  \\
\hline
\emph{Explanation:} & \multicolumn{5}{|p{12cm}|}{Integer array which indicates the entropy coder selected for coding AC components for each segment. First value is for the first segment, second value for the second segment and so on. If only one value is specified, the AC entropy coder will be the same for all segments. If only one value is specified, entropy encoder for AC components will be the same for all segments. Valid values are:\newline 0 -  No entropy code for AC components \newline 1 - CCSDS Recommended  entropy coder\newline } \\
\hline
\emph{Default:} & \multicolumn{5}{|p{12cm}|}{1 } \\
\hline
\end{tabular}\end{center}
\begin{center}\begin{tabular}{|rr|rlrl|}
\hline
\multicolumn{2}{|l|}{\textbf{$-$$-$resolutionLevels}} & \multicolumn{4}{|l|}{$\{$int$[$ int$[$ int$[$ ...$]$$]$$]$$\}$} \\
\cline{3-6}
\multicolumn{2}{|l|}{\textbf{$-$rl}} & \emph{Mandatory:} & No & &  \\
\hline
\emph{Explanation:} & \multicolumn{5}{|p{12cm}|}{Highest resolution level to be coded for each image component. 0 means that only LL subband will be compressed, 1 is LL + (HL1 + LH1 + HH1), 2 is LL + (HL1 + LH1 + HH1) + (HL2 + LH2 + HH2) and so on. \newline Values must be between 0 to WT levels. If a value is greater than the number of WT levels it will be understood that all resolution levels of the component are to be compressed. If only one value is specified, it will be used for all image components.} \\
\hline
\emph{Default:} & \multicolumn{5}{|p{12cm}|}{same of WT levels specified} \\
\hline
\end{tabular}\end{center}
\begin{center}\begin{tabular}{|rr|rlrl|}
\hline
\multicolumn{2}{|l|}{\textbf{$-$$-$pixelBitDepth}} & \multicolumn{4}{|l|}{$\{$int$[$ int$[$ int$[$ ...$]$$]$$]$$\}$} \\
\cline{3-6}
\multicolumn{2}{|l|}{\textbf{$-$pd}} & \emph{Mandatory:} & No & &  \\
\hline
\emph{Explanation:} & \multicolumn{5}{|p{12cm}|}{Integer array which indicates the pixel bit depth for each channel. First value is for the first channel, second value for the second channel and so on. If only one value is specified, the pixel bit depth will be the same for all channel.} \\
\hline
\emph{Default:} & \multicolumn{5}{|p{12cm}|}{ } \\
\hline
\end{tabular}\end{center}
\begin{center}\begin{tabular}{|rr|rlrl|}
\hline
\multicolumn{2}{|l|}{\textbf{$-$$-$verboseComputation}} & \multicolumn{4}{|l|}{$\{$boolean$\}$} \\
\cline{3-6}
\multicolumn{2}{|l|}{\textbf{$-$vc}} & \emph{Mandatory:} & No & &  \\
\hline
\emph{Explanation:} & \multicolumn{5}{|p{12cm}|}{Show some information about time and used memory for each compression stage. Value is a boolean: 0 indicates NO show and 1 indicates show.} \\
\hline
\emph{Default:} & \multicolumn{5}{|p{12cm}|}{0 } \\
\hline
\end{tabular}\end{center}
\begin{center}\begin{tabular}{|rr|rlrl|}
\hline
\multicolumn{2}{|l|}{\textbf{$-$$-$verboseParameters}} & \multicolumn{4}{|l|}{$\{$boolean$\}$} \\
\cline{3-6}
\multicolumn{2}{|l|}{\textbf{$-$vp}} & \emph{Mandatory:} & No & &  \\
\hline
\emph{Explanation:} & \multicolumn{5}{|p{12cm}|}{Show TER encoding parameters. Value is a boolean: 0 indicates NO show and 1 indicates show.} \\
\hline
\emph{Default:} & \multicolumn{5}{|p{12cm}|}{0 } \\
\hline
\end{tabular}\end{center}
\begin{center}\begin{tabular}{|rr|rlrl|}
\hline
\multicolumn{2}{|l|}{\textbf{$-$$-$bitsPerSample}} & \multicolumn{4}{|l|}{$\{$float$[$float$[$float$[$ ...$]$$]$$]$$\}$} \\
\cline{3-6}
\multicolumn{2}{|l|}{\textbf{$-$bps}} & \emph{Mandatory:} & No & &  \\
\hline
\emph{Explanation:} & \multicolumn{5}{|p{12cm}|}{Float array which indicates the bits per sample of each encoded segment. First value is for the first segment, second value for the second segment and so on. If only one value is specified, the bits per sample will be the same for all segments.} \\
\hline
\emph{Default:} & \multicolumn{5}{|p{12cm}|}{ } \\
\hline
\end{tabular}\end{center}
\begin{center}\begin{tabular}{|rr|rlrl|}
\hline
\multicolumn{2}{|l|}{\textbf{$-$$-$compressionFactor}} & \multicolumn{4}{|l|}{$\{$float$[$ float$[$ float$[$ ...$]$$]$$]$$\}$} \\
\cline{3-6}
\multicolumn{2}{|l|}{\textbf{$-$cf}} & \emph{Mandatory:} & No & &  \\
\hline
\emph{Explanation:} & \multicolumn{5}{|p{12cm}|}{Float array which indicates the compression factor for each segment. First value is for the first segment, second value for the second segment and so on. If only one value is specified, the compression factor will be the same for all segments.} \\
\hline
\emph{Default:} & \multicolumn{5}{|p{12cm}|}{1.0 } \\
\hline
\end{tabular}\end{center}
\begin{center}\begin{tabular}{|rr|rlrl|}
\hline
\multicolumn{2}{|l|}{\textbf{$-$$-$truncationPoints}} & \multicolumn{4}{|l|}{$\{$int$[$ int$[$ int$[$ ...$]$$]$$]$$\}$} \\
\cline{3-6}
\multicolumn{2}{|l|}{\textbf{$-$tp}} & \emph{Mandatory:} & No & &  \\
\hline
\emph{Explanation:} & \multicolumn{5}{|p{12cm}|}{For each segment specifies the available truncation points. First value is for the first segment, second value for the second segment and so on. If only one value is specified, the truncation points will be the same for all segments. Valid values are:\newline 0 - any \newline 1 - end of bitplane (bitPlaneStop) \newline 2 - end of resolution level (stageStop) \newline 3 - end of gaggle} \\
\hline
\emph{Default:} & \multicolumn{5}{|p{12cm}|}{0 } \\
\hline
\end{tabular}\end{center}
\begin{center}\begin{tabular}{|rr|rlrl|}
\hline
\multicolumn{2}{|l|}{\textbf{$-$$-$adjustHeaderParameters}} & \multicolumn{4}{|l|}{$\{$int$[$ int$[$ int$[$ ...$]$$]$$]$$\}$} \\
\cline{3-6}
\multicolumn{2}{|l|}{\textbf{$-$ap}} & \emph{Mandatory:} & No & &  \\
\hline
\emph{Explanation:} & \multicolumn{5}{|p{12cm}|}{For each segment specifies the parameters that should be adjusted. First value is for the first segment, second value for the second segment and so on. If only one value is specified, the asjusted parameters will be the same for all segments. Valid values are:\newline 0 - No adjustement  \newline 1 - Adjust segByteLimit and useFill after interleaving } \\
\hline
\emph{Default:} & \multicolumn{5}{|p{12cm}|}{0 } \\
\hline
\end{tabular}\end{center}
\begin{center}\begin{tabular}{|rr|rlrl|}
\hline
\multicolumn{2}{|l|}{\textbf{$-$$-$progressionOrder}} & \multicolumn{4}{|l|}{$\{$int$\}$} \\
\cline{3-6}
\multicolumn{2}{|l|}{\textbf{$-$po}} & \emph{Mandatory:} & No & &  \\
\hline
\emph{Explanation:} & \multicolumn{5}{|p{12cm}|}{This parameter specifies the progression order employed to create the encode image. Valid values are: \newline 0 Recommended order (i.e. order defined in the CCSDS recomendation for image data coding) \newline 1 LRCP Layer-Resolution-Component-Position \newline 2 RLCP Resolution-Layer-Component-Position \newline 3 RPCL Resolution-Position-Component-Layer \newline 4 PCRL Position-Component-Resolution-Layer \newline 5 CPRL Component-Position-Resolution-Layer} \\
\hline
\emph{Default:} & \multicolumn{5}{|p{12cm}|}{0 } \\
\hline
\end{tabular}\end{center}
\begin{center}\begin{tabular}{|rr|rlrl|}
\hline
\multicolumn{2}{|l|}{\textbf{$-$$-$shiftType}} & \multicolumn{4}{|l|}{$\{$int$\}$} \\
\cline{3-6}
\multicolumn{2}{|l|}{\textbf{$-$st}} & \emph{Mandatory:} & No & &  \\
\hline
\emph{Explanation:} & \multicolumn{5}{|p{12cm}|}{Level shift is a preprocessing operation for unsigned channels to convert them in signed channels. It can be performed in some diferent ways:\newline 0- No level shift\newline 1- JPEG2000 standard level shifting (only non-negative channels)\newline 2- Range center substract\newline 3- Average substract\newline 4- Specific values substract (see "-sv" parameter)} \\
\hline
\emph{Default:} & \multicolumn{5}{|p{12cm}|}{0 } \\
\hline
\end{tabular}\end{center}
\begin{center}\begin{tabular}{|rr|rlrl|}
\hline
\multicolumn{2}{|l|}{\textbf{$-$$-$shiftChannels}} & \multicolumn{4}{|l|}{$\{$boolean$[$ boolean$[$ boolean$[$ ...$]$$]$$]$$\}$} \\
\cline{3-6}
\multicolumn{2}{|l|}{\textbf{$-$sc}} & \emph{Mandatory:} & No & &  \\
\hline
\emph{Explanation:} & \multicolumn{5}{|p{12cm}|}{Specification in which channels level shift will be applied. Each value specifies a channel (0 is the first image channel).} \\
\hline
\emph{Default:} & \multicolumn{5}{|p{12cm}|}{1 } \\
\hline
\end{tabular}\end{center}
\begin{center}\begin{tabular}{|rr|rlrl|}
\hline
\multicolumn{2}{|l|}{\textbf{$-$$-$shiftValues}} & \multicolumn{4}{|l|}{$\{$int$[$ int$[$ int$[$ ...$]$$]$$]$$\}$} \\
\cline{3-6}
\multicolumn{2}{|l|}{\textbf{$-$sv}} & \emph{Mandatory:} & No & &  \\
\hline
\emph{Explanation:} & \multicolumn{5}{|p{12cm}|}{Substracted values in each image channel. If only one is specified, the value will be used for all image channel otherwise first value is for the first component, second value is for the second component, and so on.} \\
\hline
\emph{Default:} & \multicolumn{5}{|p{12cm}|}{null} \\
\hline
\end{tabular}\end{center}
\begin{center}\begin{tabular}{|rr|rlrl|}
\hline
\multicolumn{2}{|l|}{\textbf{$-$$-$coefficientsApproximation}} & \multicolumn{4}{|l|}{$\{$int$[$ int$[$ int$[$ ...$]$$]$$]$$\}$} \\
\cline{3-6}
\multicolumn{2}{|l|}{\textbf{$-$ca}} & \emph{Mandatory:} & No & &  \\
\hline
\emph{Explanation:} & \multicolumn{5}{|p{12cm}|}{This parameter specifies the approximation to be applied to coefficients of each channel. First value is for the first channel, second value for the second channel and so on. If only one value is specified, it will be the same for all channels. Valid values are:\newline 0 - Cast to integer  \newline 1 - Round  \newline 2 - Floor  \newline 3 - Ceil } \\
\hline
\emph{Default:} & \multicolumn{5}{|p{12cm}|}{1 } \\
\hline
\end{tabular}\end{center}
\begin{center}\begin{tabular}{|rr|rlrl|}
\hline
\multicolumn{2}{|l|}{\textbf{$-$$-$targetBytes}} & \multicolumn{4}{|l|}{$\{$int$\}$} \\
\cline{3-6}
\multicolumn{2}{|l|}{\textbf{$-$tb}} & \emph{Mandatory:} & No & &  \\
\hline
\emph{Explanation:} & \multicolumn{5}{|p{12cm}|}{Number of target bytes for the compressed bitstream. Computations are done without taking into account headings, so file size will be slightly larger than this number of bytes (0 indicates that size file is not important -all compressed data will be stored-).} \\
\hline
\emph{Default:} & \multicolumn{5}{|p{12cm}|}{} \\
\hline
\end{tabular}\end{center}
\begin{center}\begin{tabular}{|rr|rlrl|}
\hline
\multicolumn{2}{|l|}{\textbf{$-$$-$layerCreationType}} & \multicolumn{4}{|l|}{$\{$int$\}$} \\
\cline{3-6}
\multicolumn{2}{|l|}{\textbf{$-$lc}} & \emph{Mandatory:} & No & &  \\
\hline
\emph{Explanation:} & \multicolumn{5}{|p{12cm}|}{This parameter specifies the method used to create the layers. Valid values are:\newline 0 - One single layer is created \newline 1 - Size of layers correspond to the maximum size available \newline 2 - Coding Passes Interleaving  \newline 3 - Bitplane Interleaving} \\
\hline
\emph{Default:} & \multicolumn{5}{|p{12cm}|}{3 } \\
\hline
\end{tabular}\end{center}
\begin{center}\begin{tabular}{|rr|rlrl|}
\hline
\multicolumn{2}{|l|}{\textbf{$-$$-$numberOfLayers}} & \multicolumn{4}{|l|}{$\{$int$\}$} \\
\cline{3-6}
\multicolumn{2}{|l|}{\textbf{$-$nl}} & \emph{Mandatory:} & No & &  \\
\hline
\emph{Explanation:} & \multicolumn{5}{|p{12cm}|}{Number of layers desired by the user. If target does not allow this number, it is possible to create a file with less layers.} \\
\hline
\emph{Default:} & \multicolumn{5}{|p{12cm}|}{1 } \\
\hline
\end{tabular}\end{center}
\begin{center}\begin{tabular}{|rr|rlrl|}
\hline
\multicolumn{2}{|l|}{\textbf{$-$$-$layerSizeType}} & \multicolumn{4}{|l|}{$\{$int$\}$} \\
\cline{3-6}
\multicolumn{2}{|l|}{\textbf{$-$ls}} & \emph{Mandatory:} & No & &  \\
\hline
\emph{Explanation:} & \multicolumn{5}{|p{12cm}|}{This parameter indicates the technique employed to select the size of the layers.Valid values are:\newline 0 - Given by the user  \newline 1 - All layers have the same length \newline 2 - Each layer has half of the length that the next layer (except for the first).} \\
\hline
\emph{Default:} & \multicolumn{5}{|p{12cm}|}{2 } \\
\hline
\end{tabular}\end{center}
\begin{center}\begin{tabular}{|rr|rlrl|}
\hline
\multicolumn{2}{|l|}{\textbf{$-$$-$layerBytes}} & \multicolumn{4}{|l|}{$\{$int$[$ int$[$ int$[$ ...$]$$]$$]$$\}$} \\
\cline{3-6}
\multicolumn{2}{|l|}{\textbf{$-$lb}} & \emph{Mandatory:} & No & &  \\
\hline
\emph{Explanation:} & \multicolumn{5}{|p{12cm}|}{This parameters indicates the size of each layer. Each value specifies a layer, if there appear less values than layers demanded, the last value is taken as a default value for the rest of layers.} \\
\hline
\emph{Default:} & \multicolumn{5}{|p{12cm}|}{ } \\
\hline
\end{tabular}\end{center}
\begin{center}\begin{tabular}{|rr|rlrl|}
\hline
\multicolumn{2}{|l|}{\textbf{$-$$-$bitsPerPixel}} & \multicolumn{4}{|l|}{$\{$float$\}$} \\
\cline{3-6}
\multicolumn{2}{|l|}{\textbf{$-$bpp}} & \emph{Mandatory:} & No & &  \\
\hline
\emph{Explanation:} & \multicolumn{5}{|p{12cm}|}{Float which indicates the bits per sample that must be used to encode the image.} \\
\hline
\emph{Default:} & \multicolumn{5}{|p{12cm}|}{ } \\
\hline
\end{tabular}\end{center}
\begin{center}\begin{tabular}{|rr|rlrl|}
\hline
\multicolumn{2}{|l|}{\textbf{$-$$-$test3d}} & \multicolumn{4}{|l|}{$\{$int$\}$} \\
\cline{3-6}
\multicolumn{2}{|l|}{\textbf{$-$t3}} & \emph{Mandatory:} & No & &  \\
\hline
\emph{Explanation:} & \multicolumn{5}{|p{12cm}|}{Kind of 3D Discrete wavelet transform to be applied. Valid values are:\newline 0 - 2D DWT in the spatial domain  \newline 1 - 1D + 2D hybrid  DWT \newline 2 - 3D pyramidal DWT} \\
\hline
\emph{Default:} & \multicolumn{5}{|p{12cm}|}{ } \\
\hline
\end{tabular}\end{center}
\begin{center}\begin{tabular}{|rr|rlrl|}
\hline
\multicolumn{2}{|l|}{\textbf{$-$$-$spectralWTLevels}} & \multicolumn{4}{|l|}{$\{$int$\}$} \\
\cline{3-6}
\multicolumn{2}{|l|}{\textbf{$-$swl}} & \emph{Mandatory:} & No & &  \\
\hline
\emph{Explanation:} & \multicolumn{5}{|p{12cm}|}{Discrete wavelet transform levels to be applied in the spatial domain.} \\
\hline
\emph{Default:} & \multicolumn{5}{|p{12cm}|}{ } \\
\hline
\end{tabular}\end{center}
\begin{center}\begin{tabular}{|rr|rlrl|}
\hline
\multicolumn{2}{|l|}{\textbf{$-$$-$spectralWTType}} & \multicolumn{4}{|l|}{$\{$int$\}$} \\
\cline{3-6}
\multicolumn{2}{|l|}{\textbf{$-$swt}} & \emph{Mandatory:} & No & &  \\
\hline
\emph{Explanation:} & \multicolumn{5}{|p{12cm}|}{Discrete wavelet transform type to be applied in the spatial domain.} \\
\hline
\emph{Default:} & \multicolumn{5}{|p{12cm}|}{ } \\
\hline
\end{tabular}\end{center}


\newpage

\subsection{TERdecoder parameters}
\label{sect:decoderParameters}

\begin{center}\begin{tabular}{|rr|rlrl|}
\hline
\multicolumn{2}{|l|}{\textbf{$-$$-$help}} & \multicolumn{4}{|l|}{} \\
\cline{3-6}
\multicolumn{2}{|l|}{\textbf{$-$h}} & \emph{Mandatory:} & No & &  \\
\hline
\emph{Explanation:} & \multicolumn{5}{|p{12cm}|}{Displays this help and exits program.} \\
\hline
\emph{Default:} & \multicolumn{5}{|p{12cm}|}{} \\
\hline
\end{tabular}\end{center}
\begin{center}\begin{tabular}{|rr|rlrl|}
\hline
\multicolumn{2}{|l|}{\textbf{$-$$-$outputImage}} & \multicolumn{4}{|l|}{$\{$string$\}$} \\
\cline{3-6}
\multicolumn{2}{|l|}{\textbf{$-$o}} & \emph{Mandatory:} & Yes & &  \\
\hline
\emph{Explanation:} & \multicolumn{5}{|p{12cm}|}{Decoded image. Valid formats are: pgm, ppm, pbm, jpg, tiff, png, bmp, gif, fpx. If image is raw data file extension must be ".raw" and "-g" parameter is mandatory.} \\
\hline
\emph{Default:} & \multicolumn{5}{|p{12cm}|}{} \\
\hline
\end{tabular}\end{center}
\begin{center}\begin{tabular}{|rr|rlrl|}
\hline
\multicolumn{2}{|l|}{\textbf{$-$$-$imageGeometry}} & \multicolumn{4}{|l|}{$\{$int int int int int boolean$\}$} \\
\cline{3-6}
\multicolumn{2}{|l|}{\textbf{$-$g}} & \emph{Mandatory:} & No & &  \\
\hline
\emph{Explanation:} & \multicolumn{5}{|p{12cm}|}{Geometry of raw image data. Parameters are:\newline 1- zSize (number of image components)\newline 2- ySize (image height)\newline 3- xSize (image width)\newline 4- data type. Possible values are:\newline \hspace*{0.5truecm} 0- boolean (1 byte)\newline \hspace*{0.5truecm} 1- unsigned int (1 byte)\newline \hspace*{0.5truecm} 2- unsigned int (2 bytes)\newline \hspace*{0.5truecm} 3- signed int (2 bytes)\newline \hspace*{0.5truecm} 4- signed int (4 bytes)\newline \hspace*{0.5truecm} 5- signed int (8 bytes)\newline \hspace*{0.5truecm} 6- float (4 bytes)\newline \hspace*{0.5truecm} 7- double (8 bytes)\newline 5- Byte order (0 if BIG ENDIAN, 1 if LITTLE ENDIAN)\newline 6- 1 if 3 first components are RGB, 0 otherwise.} \\
\hline
\emph{Default:} & \multicolumn{5}{|p{12cm}|}{} \\
\hline
\end{tabular}\end{center}
\begin{center}\begin{tabular}{|rr|rlrl|}
\hline
\multicolumn{2}{|l|}{\textbf{$-$$-$inputFile}} & \multicolumn{4}{|l|}{$\{$string$\}$} \\
\cline{3-6}
\multicolumn{2}{|l|}{\textbf{$-$i}} & \emph{Mandatory:} & Yes & &  \\
\hline
\emph{Explanation:} & \multicolumn{5}{|p{12cm}|}{Input encoded image file name.} \\
\hline
\emph{Default:} & \multicolumn{5}{|p{12cm}|}{} \\
\hline
\end{tabular}\end{center}
\begin{center}\begin{tabular}{|rr|rlrl|}
\hline
\multicolumn{2}{|l|}{\textbf{$-$$-$segByteLimit}} & \multicolumn{4}{|l|}{$\{$int$[$ int$[$ int$[$ ...$]$$]$$]$$\}$} \\
\cline{3-6}
\multicolumn{2}{|l|}{\textbf{$-$bl}} & \emph{Mandatory:} & No & &  \\
\hline
\emph{Explanation:} & \multicolumn{5}{|p{12cm}|}{Maximum number of bytes that can be used in a segment.\newline The value of SegByteLimits includes bytes used for the header, and is applied to the subsequent segments until a new value is given. Value should be in the interval (0 , 134217728]} \\
\hline
\emph{Default:} & \multicolumn{5}{|p{12cm}|}{134217728 } \\
\hline
\end{tabular}\end{center}
\begin{center}\begin{tabular}{|rr|rlrl|}
\hline
\multicolumn{2}{|l|}{\textbf{$-$$-$DCStop}} & \multicolumn{4}{|l|}{$\{$int$[$ int$[$ int$[$ ...$]$$]$$]$$\}$} \\
\cline{3-6}
\multicolumn{2}{|l|}{\textbf{$-$dc}} & \emph{Mandatory:} & No & &  \\
\hline
\emph{Explanation:} & \multicolumn{5}{|p{12cm}|}{Indicates whether compressed output stops after coding of quantized DC coefficients.  If only one value is specified, type will be used for all image segments.} \\
\hline
\emph{Default:} & \multicolumn{5}{|p{12cm}|}{0 } \\
\hline
\end{tabular}\end{center}
\begin{center}\begin{tabular}{|rr|rlrl|}
\hline
\multicolumn{2}{|l|}{\textbf{$-$$-$bitPlaneStop}} & \multicolumn{4}{|l|}{$\{$int$[$ int$[$ int$[$ ...$]$$]$$]$$\}$} \\
\cline{3-6}
\multicolumn{2}{|l|}{\textbf{$-$bp}} & \emph{Mandatory:} & No & &  \\
\hline
\emph{Explanation:} & \multicolumn{5}{|p{12cm}|}{Only used when DCStop equals 0. Indicates the bit plane index where the codification should stop.  If only one value is specified, type will be used for all image segments.} \\
\hline
\emph{Default:} & \multicolumn{5}{|p{12cm}|}{0 } \\
\hline
\end{tabular}\end{center}
\begin{center}\begin{tabular}{|rr|rlrl|}
\hline
\multicolumn{2}{|l|}{\textbf{$-$$-$stageStop}} & \multicolumn{4}{|l|}{$\{$int$[$ int$[$ int$[$ ...$]$$]$$]$$\}$} \\
\cline{3-6}
\multicolumn{2}{|l|}{\textbf{$-$ss}} & \emph{Mandatory:} & No & &  \\
\hline
\emph{Explanation:} & \multicolumn{5}{|p{12cm}|}{Indicates the stage where the codification ends.  If only one value is specified, type will be used for all image segments. If three levels of DWT are applied, then: \newline 1 - stage 1 \newline 2 - stage 2 \newline 3 - stage 3 \newline 4 - stage 4 } \\
\hline
\emph{Default:} & \multicolumn{5}{|p{12cm}|}{The same of wavelet levels plus one} \\
\hline
\end{tabular}\end{center}
\begin{center}\begin{tabular}{|rr|rlrl|}
\hline
\multicolumn{2}{|l|}{\textbf{$-$$-$verboseComputation}} & \multicolumn{4}{|l|}{$\{$boolean$\}$} \\
\cline{3-6}
\multicolumn{2}{|l|}{\textbf{$-$vc}} & \emph{Mandatory:} & No & &  \\
\hline
\emph{Explanation:} & \multicolumn{5}{|p{12cm}|}{Show some information about time and used memory for each compression stage. Value is a boolean: 0 indicates NO show and 1 indicates show.} \\
\hline
\emph{Default:} & \multicolumn{5}{|p{12cm}|}{0 } \\
\hline
\end{tabular}\end{center}
\begin{center}\begin{tabular}{|rr|rlrl|}
\hline
\multicolumn{2}{|l|}{\textbf{$-$$-$verboseParameters}} & \multicolumn{4}{|l|}{$\{$boolean$\}$} \\
\cline{3-6}
\multicolumn{2}{|l|}{\textbf{$-$vp}} & \emph{Mandatory:} & No & &  \\
\hline
\emph{Explanation:} & \multicolumn{5}{|p{12cm}|}{Show TER encoding parameters. Value is a boolean: 0 indicates NO show and 1 indicates show.} \\
\hline
\emph{Default:} & \multicolumn{5}{|p{12cm}|}{0 } \\
\hline
\end{tabular}\end{center}
\begin{center}\begin{tabular}{|rr|rlrl|}
\hline
\multicolumn{2}{|l|}{\textbf{$-$$-$verboseMessages}} & \multicolumn{4}{|l|}{$\{$boolean$\}$} \\
\cline{3-6}
\multicolumn{2}{|l|}{\textbf{$-$vm}} & \emph{Mandatory:} & No & &  \\
\hline
\emph{Explanation:} & \multicolumn{5}{|p{12cm}|}{Show TER decoder messages, for instance if the file ends unexpectedly. Value is a boolean: 0 indicates NO show and 1 indicates show.} \\
\hline
\emph{Default:} & \multicolumn{5}{|p{12cm}|}{0 } \\
\hline
\end{tabular}\end{center}
\begin{center}\begin{tabular}{|rr|rlrl|}
\hline
\multicolumn{2}{|l|}{\textbf{$-$$-$gammaValue}} & \multicolumn{4}{|l|}{$\{$float$[$ float$[$ float$[$ ...$]$$]$$]$$\}$} \\
\cline{3-6}
\multicolumn{2}{|l|}{\textbf{$-$gv}} & \emph{Mandatory:} & No & &  \\
\hline
\emph{Explanation:} & \multicolumn{5}{|p{12cm}|}{Float array that indicates for each channel the way of recovering the values. First value is for the first channel, second value for the second channel and so on. If only one value is specified, it will be the same for all channels.\newline This float indicates the point of the interval where values are recovered in the decoding process. When decoding a value inside an interval is possible to approximate this value to the middle value of the interval, to the low value, to 0,25 of the interval, to ... \newline Example: in bit plane 3, a recovered value can be approximated to 8 (low value) or to 12 (middle value). \newline Valid values are real numbers in the interval [0,1). \newline For example: \newline   0.0 - Lower value (gamma = 0) \newline   0.5 - Middle value (gamma = 1/2) \newline   0.375 - 3/8 of the threshold value (gamma = 3/8)} \\
\hline
\emph{Default:} & \multicolumn{5}{|p{12cm}|}{0.375 } \\
\hline
\end{tabular}\end{center}
\begin{center}\begin{tabular}{|rr|rlrl|}
\hline
\multicolumn{2}{|l|}{\textbf{$-$$-$completionMode}} & \multicolumn{4}{|l|}{$\{$int$[$ int$[$ int$[$ ...$]$$]$$]$$\}$} \\
\cline{3-6}
\multicolumn{2}{|l|}{\textbf{$-$cm}} & \emph{Mandatory:} & No & &  \\
\hline
\emph{Explanation:} & \multicolumn{5}{|p{12cm}|}{Integer array that indicates for each channel the way of completing DC values when the bitstream ends unexpectedly. First value is for the first channel, second value for the second channel and so on. If only one value is specified, it will be the same for all channels. \newline Valid values are:\newline 0 - Add zeros \newline 1 - Repeat the middle value of the magnitude interval where the components are being decoded, i.e., (max(-xMin,xMax)+1)/2. \newline 2 - Repeat the mean of the decoded values \newline n - For n>2, repeat the mean value of the n-2 last recovered values. If less than n-2 values have been recovered, only the recovered values will be used. Note that for n=3 we repeat the last value.} \\
\hline
\emph{Default:} & \multicolumn{5}{|p{12cm}|}{0 } \\
\hline
\end{tabular}\end{center}
\begin{center}\begin{tabular}{|rr|rlrl|}
\hline
\multicolumn{2}{|l|}{\textbf{$-$$-$minusHalf}} & \multicolumn{4}{|l|}{$\{$boolean$\}$} \\
\cline{3-6}
\multicolumn{2}{|l|}{\textbf{$-$mh}} & \emph{Mandatory:} & No & &  \\
\hline
\emph{Explanation:} & \multicolumn{5}{|p{12cm}|}{Boolean value that indicated if the user wants to substract 0.5 to the recovered values in the bit plane encoder, as Recomendation does for the lossy case.} \\
\hline
\emph{Default:} & \multicolumn{5}{|p{12cm}|}{false } \\
\hline
\end{tabular}\end{center}
\begin{center}\begin{tabular}{|rr|rlrl|}
\hline
\multicolumn{2}{|l|}{\textbf{$-$$-$test3d}} & \multicolumn{4}{|l|}{$\{$int$\}$} \\
\cline{3-6}
\multicolumn{2}{|l|}{\textbf{$-$t3}} & \emph{Mandatory:} & No & &  \\
\hline
\emph{Explanation:} & \multicolumn{5}{|p{12cm}|}{Kind of 3D Discrete wavelet transform to be applied. Valid values are:\newline 0 - 2D DWT in the spatial domain  \newline 1 - 1D + 2D hybrid  DWT \newline 2 - 3D pyramidal DWT} \\
\hline
\emph{Default:} & \multicolumn{5}{|p{12cm}|}{ } \\
\hline
\end{tabular}\end{center}
\begin{center}\begin{tabular}{|rr|rlrl|}
\hline
\multicolumn{2}{|l|}{\textbf{$-$$-$spectralWTLevels}} & \multicolumn{4}{|l|}{$\{$int$\}$} \\
\cline{3-6}
\multicolumn{2}{|l|}{\textbf{$-$swl}} & \emph{Mandatory:} & No & &  \\
\hline
\emph{Explanation:} & \multicolumn{5}{|p{12cm}|}{Discrete wavelet transform levels to be applied in the spatial domain.} \\
\hline
\emph{Default:} & \multicolumn{5}{|p{12cm}|}{ } \\
\hline
\end{tabular}\end{center}
\begin{center}\begin{tabular}{|rr|rlrl|}
\hline
\multicolumn{2}{|l|}{\textbf{$-$$-$spectralWTType}} & \multicolumn{4}{|l|}{$\{$int$\}$} \\
\cline{3-6}
\multicolumn{2}{|l|}{\textbf{$-$swt}} & \emph{Mandatory:} & No & &  \\
\hline
\emph{Explanation:} & \multicolumn{5}{|p{12cm}|}{Discrete wavelet transform type to be applied in the spatial domain.} \\
\hline
\emph{Default:} & \multicolumn{5}{|p{12cm}|}{ } \\
\hline
\end{tabular}\end{center}


\newpage

\subsection{TERdisplay parameters}
\label{sect:displayParameters}

\begin{center}\begin{tabular}{|rr|rlrl|}
\hline
\multicolumn{2}{|l|}{\textbf{$-$$-$help}} & \multicolumn{4}{|l|}{} \\
\cline{3-6}
\multicolumn{2}{|l|}{\textbf{$-$h}} & \emph{Mandatory:} & No & &  \\
\hline
\emph{Explanation:} & \multicolumn{5}{|p{12cm}|}{1} \\
\hline
\emph{Default:} & \multicolumn{5}{|p{12cm}|}{} \\
\hline
\end{tabular}\end{center}
\begin{center}\begin{tabular}{|rr|rlrl|}
\hline
\multicolumn{2}{|l|}{\textbf{$-$$-$inputFile}} & \multicolumn{4}{|l|}{$\{$string$\}$} \\
\cline{3-6}
\multicolumn{2}{|l|}{\textbf{$-$i}} & \emph{Mandatory:} & Yes & &  \\
\hline
\emph{Explanation:} & \multicolumn{5}{|p{12cm}|}{1} \\
\hline
\emph{Default:} & \multicolumn{5}{|p{12cm}|}{} \\
\hline
\end{tabular}\end{center}
\begin{center}\begin{tabular}{|rr|rlrl|}
\hline
\multicolumn{2}{|l|}{\textbf{$-$$-$resolutionLevels}} & \multicolumn{4}{|l|}{$\{$int$\}$} \\
\cline{3-6}
\multicolumn{2}{|l|}{\textbf{$-$rl}} & \emph{Mandatory:} & No & &  \\
\hline
\emph{Explanation:} & \multicolumn{5}{|p{12cm}|}{0} \\
\hline
\emph{Default:} & \multicolumn{5}{|p{12cm}|}{1} \\
\hline
\end{tabular}\end{center}
\begin{center}\begin{tabular}{|rr|rlrl|}
\hline
\multicolumn{2}{|l|}{\textbf{$-$$-$numberOfLayers}} & \multicolumn{4}{|l|}{$\{$int$\}$} \\
\cline{3-6}
\multicolumn{2}{|l|}{\textbf{$-$nl}} & \emph{Mandatory:} & No & &  \\
\hline
\emph{Explanation:} & \multicolumn{5}{|p{12cm}|}{0} \\
\hline
\emph{Default:} & \multicolumn{5}{|p{12cm}|}{1 } \\
\hline
\end{tabular}\end{center}
\begin{center}\begin{tabular}{|rr|rlrl|}
\hline
\multicolumn{2}{|l|}{\textbf{$-$$-$targetBytes}} & \multicolumn{4}{|l|}{$\{$int$\}$} \\
\cline{3-6}
\multicolumn{2}{|l|}{\textbf{$-$tb}} & \emph{Mandatory:} & No & &  \\
\hline
\emph{Explanation:} & \multicolumn{5}{|p{12cm}|}{0} \\
\hline
\emph{Default:} & \multicolumn{5}{|p{12cm}|}{ } \\
\hline
\end{tabular}\end{center}
\begin{center}\begin{tabular}{|rr|rlrl|}
\hline
\multicolumn{2}{|l|}{\textbf{$-$$-$channelList}} & \multicolumn{4}{|l|}{$\{$int$[$ int$[$ int$[$ ...$]$$]$$]$$\}$} \\
\cline{3-6}
\multicolumn{2}{|l|}{\textbf{$-$cl}} & \emph{Mandatory:} & No & &  \\
\hline
\emph{Explanation:} & \multicolumn{5}{|p{12cm}|}{0} \\
\hline
\emph{Default:} & \multicolumn{5}{|p{12cm}|}{ } \\
\hline
\end{tabular}\end{center}
\begin{center}\begin{tabular}{|rr|rlrl|}
\hline
\multicolumn{2}{|l|}{\textbf{$-$$-$yInit}} & \multicolumn{4}{|l|}{$\{$int$\}$} \\
\cline{3-6}
\multicolumn{2}{|l|}{\textbf{$-$y0}} & \emph{Mandatory:} & No & &  \\
\hline
\emph{Explanation:} & \multicolumn{5}{|p{12cm}|}{0} \\
\hline
\emph{Default:} & \multicolumn{5}{|p{12cm}|}{ } \\
\hline
\end{tabular}\end{center}
\begin{center}\begin{tabular}{|rr|rlrl|}
\hline
\multicolumn{2}{|l|}{\textbf{$-$$-$yLength}} & \multicolumn{4}{|l|}{$\{$int$\}$} \\
\cline{3-6}
\multicolumn{2}{|l|}{\textbf{$-$yl}} & \emph{Mandatory:} & No & &  \\
\hline
\emph{Explanation:} & \multicolumn{5}{|p{12cm}|}{0} \\
\hline
\emph{Default:} & \multicolumn{5}{|p{12cm}|}{ } \\
\hline
\end{tabular}\end{center}
\begin{center}\begin{tabular}{|rr|rlrl|}
\hline
\multicolumn{2}{|l|}{\textbf{$-$$-$xInit}} & \multicolumn{4}{|l|}{$\{$int$\}$} \\
\cline{3-6}
\multicolumn{2}{|l|}{\textbf{$-$x0}} & \emph{Mandatory:} & No & &  \\
\hline
\emph{Explanation:} & \multicolumn{5}{|p{12cm}|}{0} \\
\hline
\emph{Default:} & \multicolumn{5}{|p{12cm}|}{ } \\
\hline
\end{tabular}\end{center}
\begin{center}\begin{tabular}{|rr|rlrl|}
\hline
\multicolumn{2}{|l|}{\textbf{$-$$-$xLength}} & \multicolumn{4}{|l|}{$\{$int$\}$} \\
\cline{3-6}
\multicolumn{2}{|l|}{\textbf{$-$xl}} & \emph{Mandatory:} & No & &  \\
\hline
\emph{Explanation:} & \multicolumn{5}{|p{12cm}|}{0} \\
\hline
\emph{Default:} & \multicolumn{5}{|p{12cm}|}{ } \\
\hline
\end{tabular}\end{center}
\begin{center}\begin{tabular}{|rr|rlrl|}
\hline
\multicolumn{2}{|l|}{\textbf{$-$$-$extractionType}} & \multicolumn{4}{|l|}{$\{$int$\}$} \\
\cline{3-6}
\multicolumn{2}{|l|}{\textbf{$-$et}} & \emph{Mandatory:} & No & &  \\
\hline
\emph{Explanation:} & \multicolumn{5}{|p{12cm}|}{0} \\
\hline
\emph{Default:} & \multicolumn{5}{|p{12cm}|}{ } \\
\hline
\end{tabular}\end{center}


\newpage
%%%%%%%%%%%%%%%%%%%%%%%%%%%%%%%%%%%%%%%%%%%%%%%%%%%%%%%%%%%%%%%%%%%%%%%%%%%%%%%
\section{Examples}
\label{sect:examples}

\vspace{0.25cm}\textbf{LOSSLESS COMPRESSION USING A SINGLE SEGMENT}\vspace{0.25cm}  

\emph{\#TERcode -i workDir/mars512x32.raw -g 1 32 512 1 0 0 -of 0
  workDir/mars512x32} \\

This execution will generate a file (compilant with the
Recommendation) called \emph{mars512x32.rec}. The default mode of TER
is lossless compression using a single segment and employing part 2,3
and 4 flag for the segment. To decompress the file and recover the
original image run:

\emph{\#TERdecode -i workDir/mars512x32.rec -o
  workDir/mars512x32.ter.raw -g 1 32 512 1 0 0 } \\   

If the user wants to store the image in a known format, for example
'pgm' format, then there is not needed the geometry of the output
image. The user could run:

\emph{\# TERdecode -i workDir/mars512x32.rec -o
  workDir/mars512x32.ter.pgm} \\ 

\vspace{0.25cm}\textbf{LOSSLESS COMPRESSION USING MORE THAN ONE
  SEGMENT}\vspace{0.25cm}  
		
\emph{\#TERcode -i workDir/sar16bit.raw -g 1 512 512 2 0 0 -wt 4
  --part2Flag 1 0 --part3Flag 1 0 --part4Flag 1 0 --blocksPerSegment
  64 -of 0} \\    
		
This execution will generate a file (compilant with the
Recommendation) called \emph{sar16bit.raw.rec}. The wavelet employed
is the integer 9/7, each segment contains 64 blocks and only the first
segment contains the part 2, 3 and 4 header. All other segments have
only Part1 header. To decompress the file and recover the original
image run: 

\emph{\#./TERdecode -i workDir/sar16bit.raw.rec -g 1 512 512 2 0 0 -o
  workDir/sar16bit.raw.rec.raw} 


\vspace{0.25cm}\textbf{LOSSY COMPRESSION}\vspace{0.25cm}

\emph{\#TERcode -i workDir/foc.raw -g 1 512 1024 2 0 0 --pixelBitDepth
  12 - --blocksPerSegment 128 wt 3 -o workDir/foc.raw.rec -of 1}

This exection uses the float 9/7 wavelet transform, then some loss is
produce when coefficients are transformed. To decompress the file and
recover the original image run:

\emph{\#TERdecode -i workDir/foc.raw.rec -o workDir/foc.raw.rec.raw
  -g 1 512 1024 2 0 0}  


\vspace{0.25cm}\textbf{COMPRESSION USING A FIXED SEGBYTELIMIT}\vspace{0.25cm}  

\emph{\#./TERcode -i workDir/sar16bit.raw -g 1 512 512 2 0 0 -wt 4 --part2Flag
1 0 --part3Flag 1 0 --part4Flag 1 0 --blocksPerSegment 64
 -of 0 --segByteLimit 3072 --useFill 1}

This execution creates a file where each segment size is limited to
3072 bytes, in case the size of the segment in lower that 3072, it is
completed with zeros up to the required size. To decompress
the file and recover the original image run:

\emph{\#./TERdecode -i workDir/sar16bit.raw.rec -g 1 512 512 2 0 0 -o
  workDir/sar16bit.raw.rec.raw} 

\vspace{0.25cm}\textbf{COMPRESSION USING CUSTOM WEIGHTS}\vspace{0.25cm} 

\emph{\#TERcode -i workDir/marstest.raw -g 1 512 512 1 0 0 -wt 4
  --part2Flag 1 0 --part3Flag 1 0 --part4Flag 1 0 --blocksPerSegment
  64 --pixelBitDepth 8 -of 0 -wg 1 -cw 4 4 4 2 2
  2 1 1 1 1} 

This execution produces a encoded file with the scaling factor for
each subband of the transformed image given by the user. To decompress
the file and recover the original image run:

\emph{\#TERdecode -i workDir/marstest.raw.rec -o
  workDir/marstest.raw.rec.raw -g 1 512 512 1 0 0} 

\vspace{0.25cm}\textbf{ONLY DC COMPONENTS COMPRESSION}\vspace{0.25cm}
\emph{\#TERcode -i workDir/marstest.raw -g 1 512 512 1 0 0 -o
workDir/temp.coded -of 0 -wt 4 --blocksPerSegment 64 --part2Flag 1 0
--part3Flag 1 0 --part4Flag 1 0 -dc 1}

This execution produces a encoded file where only DC components are
encoded. To decompress the file and recover the original image run:

\emph{\#TERdecode -i workDir/temp.coded.rec -o
  workDir/marstest.raw.rec.raw -g 1 512 512 1 0 0} 

\vspace{0.25cm}\textbf{COMPRESSION UP TO A FIXED BITPLANE AND
  STAGE}\vspace{0.25cm}   

\emph{\#TERcode -i workDir/marstest.raw -g 1 512 512 1 0 0 -o
workDir/marstest.raw -of 0 -wt 4 --blocksPerSegment 64 --part2Flag 1 0
--part3Flag 1 0 --part4Flag 1 0 -dc 0 -bp 1 -ss 2}

This execution produces a encoded file that each segment is encoded up
to the bit plane 1 and the stage 2. To decompress
the file and recover the original image run:

\emph{\#TERdecode -i workDir/marstest.raw.rec -g 1 512 512 1 0 0 -o
workDir/marstest.raw.rec.raw}

\vspace{0.25cm}\textbf{COMPRESSION USING A DIFFERENT ANY NUMBER OF
  WAVELET LEVELS}\vspace{0.25cm}  

\emph{\#TERcode -i workDir/lena.pgm -wl 5 -wt 3 -o
  workDir/lena.pgm -of 1 -po 0}

This execution produces a file,  workDir/lena.pgm.ter,
that is not compilant with the Recommendation.
To decompress the file and recover the original image run:

\emph{\#TERcode -i workDir/lena.pgm.ter -o
  workDir/lena.pgm.ter.pgm}

\vspace{0.25cm}\textbf{SHOWING COMPRESSION
  STATISTICS}\vspace{0.25cm} 

In some cases it is interesting to show the time that each stage
lasts and the used memory. It can be done by: 

\emph{\#TERcode -i inputImage.pgm -o outputFile -vc 1} \\ 

\end{document}
