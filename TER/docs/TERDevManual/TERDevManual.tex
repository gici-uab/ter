\documentclass[a4paper,10pt]{article}
\usepackage[]{graphicx}
\usepackage[]{times}
\usepackage{geometry}

\geometry{verbose,a4paper,tmargin=1.5cm,bmargin=2cm,lmargin=2cm,rmargin=2cm}
\renewcommand{\baselinestretch}{1.2}

\title{TER development manual \\ \small (version 2.0 beta)}

\author{
GICI group \vspace{0.1cm} \\
\small Department of Information and Communications Engineering \\
\small Universitat Aut{\`o}noma Barcelona \\
\small http://www.gici.uab.es  -  http://www.gici.uab.es/TER \\
\small http://sourceforge.net/projects/ter
}

\date{November 2007}

\begin{document}
\maketitle

	The goal of this document is to provide a quick guide to developers
	that need to modify TER application in some sense. An overview of
	the TER design and the directory structures as well as the basic
	compilation instructions is explained. However, the detailed
	information about data structures, algorithms and classes is
	documented in the TER API. 

%%%%%%%%%%%%%%%%%%%%%%%%%%%%%%%%%%%%%%%%%%%%%%%%%%%%%%%%%%%%%%%%%%%%%%%%%%%%%%%
\section{Overview}
\label{sect:overview}

	The main motivation in the design and development of TER is to
	generate a completely modularized scheme where each module works
	independently. In order to understand it better, all modules have
	the same skeleton and only basic programming language tools are
	used. The main advantage of these independent modules is that one
	module can be replaced without compromising the others, easing the
	testing of new ideas, the extension on some operations, and even
	the replacement of some coding operations. 

	All compression stages are divided in simple modules that interact
	among them. Each module is programmed independently and has its own
	parameters. Replacement of some coding operations, testing new
	ideas, debugging tasks and even extending coding operations are an
	easy task with TER. The interaction between modules is performed
	with structures, passed among classes. The coder manages these
	structures and controls their destruction. 


%%%%%%%%%%%%%%%%%%%%%%%%%%%%%%%%%%%%%%%%%%%%%%%%%%%%%%%%%%%%%%%%%%%%%%%%%%%%%%%
\section{Files \& directories}
\label{sect:files}

	To organize all the classes and files, TER is organized using 
	directories. The root directory contains the following items: 

	\begin{itemize}
		\item \textbf{build}: object files used in the compilation
		(.class files) 
		\item \textbf{dist}: the encapsulated coder and decoder in .jar
		files (TERcode.jar, TERdecode.jar, and TERdisplay.jar) 
		\item \textbf{docs}: TER manuals (installation, user and
		development) and the api documentation 
		\item \textbf{workDir}: a temporal working dir that can be used
		by users 
		\item \textbf{src}: all the source codes are contained in this
		directory. It contains all the external libraries used in BOI
		(developed by the GICI group and used in other implementations)
		and the source code of TER. The libraries are in the directories
		(GiciAnalysis, GiciException, GiciFile, GiciImageExtension,
		GiciStream and GiciTransform) and the sources in TER directory.  
	\end{itemize}

	As we can see in the TER source directory contains the
	coder and the decoder sources separately. All the source code is
	structured in small classes that are called from the main coder or
	decoder (\emph{src/TER/TERcoder/Coder.java} or
	\emph{src/TER/TERdecoder/Decoder.java}). Each one of these small
	classes performs simple operations and all of them use the same
	skeleton structure, easing its comprehension and modification. In
	addition, for each class the execution processes are the same:
	first, they perform initializations, then a parametrization can be
	used, and last the run procedure executes module actions.


%%%%%%%%%%%%%%%%%%%%%%%%%%%%%%%%%%%%%%%%%%%%%%%%%%%%%%%%%%%%%%%%%%%%%%%%%%%%%%%
\section{Compilation and execution}
\label{sect:compilation}

	The ant tool is used to compile the source files. This tool is
	configured with the file \emph{build.xml} located in the root
	directory and contains all the compilation instructions. Executing
	\emph{ant compile} in this root directory all the sources contained
	in the \emph{src/} directory will be compiled and the .class files
	that this process generates will be located at the \emph{build/}
	directory. If compilation does not generate any error the .jar
	files will be generated at the \emph{dist/} directory. This jar
	files can be executed using the shell scripts \emph{TERcode} and
	\emph{TERdecode} or using the command lines \emph{java -jar
	dist/TERcode.jar} and \emph{java -jar dist/TERdecode.jar} (see the
	user manual for more information about how to run TER). 

	To clean all directories of garbage and swap files you can run
	\emph{ant run}. Moreover, the whole source code is well commented
	to facilitate its understanding and modifications. \emph{ant doc}
	generates the api documentation. This command creates the
	\emph{docs/api} directory with html that can be viewed with and
	standard browser. 

	Add, replace or even remove source files does not effect the
	compilation process. However, if you need to create or remove the
	directory structure you will have to modify the \emph{build.xml}
	file in order to indicate your changes to the compilation
	process. However, manual compilation with the \emph{javac} command
	is also possible and the compilation options are also contained in
	the \emph{build.xml} file. 


\end{document}
